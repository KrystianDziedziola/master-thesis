
% ------------------------------------------------------------------------
%   Kropki po numerach sekcji, podsekcji, itd.
%   Np. 1.2. Tytu� podrozdzia�u
% ------------------------------------------------------------------------
\makeatletter
    \def\numberline#1{\hb@xt@\@tempdima{#1.\hfil}}                      %kropki w spisie tre�ci
    \renewcommand*\@seccntformat[1]{\csname the#1\endcsname.\enspace}   %kropki w tre�ci dokumentu
\makeatother

\makeatother
% ------------------------------------------------------------------------
% Definicje
% ------------------------------------------------------------------------
\def\nonumsection#1{%
    \section*{#1}%
    \addcontentsline{toc}{section}{#1}%
    }
\def\nonumsubsection#1{%
    \subsection*{#1}%
    \addcontentsline{toc}{subsection}{#1}%
    }
\reversemarginpar %umieszcza notki po lewej stronie, czyli tam gdzie jest wi�cej miejsca
\def\notka#1{%
    \marginpar{\footnotesize{#1}}%
    }
%\def\mathcal#1{%
%    \mathscr{#1}%
%    }

\newcommand{\myemptypage}{ \newpage  \thispagestyle{empty}~\newpage}

%-------------------------------------------------------------------------
% listingi
%-------------------------------------------------------------------------
\lstdefinestyle{praca}{basicstyle=\scriptsize\ttfamily, 
								keywordstyle=\color{black}\bfseries,
								numbers=left, 
								stepnumber=1, 
								numberstyle=\tiny, 
								numbersep=10pt,
								extendedchars=true, 
								frame=tb}
\lstset{style=praca}

%-------------------------------------------------------------------------
% stopka i nag��wek
%-------------------------------------------------------------------------
\setlength{\headheight}{15pt}

\pagestyle{fancy}
\renewcommand{\chaptermark}[1]{\markboth{#1}{}}
\renewcommand{\sectionmark}[1]{\markright{#1}{}}

\fancyhf{}
\fancyhead[LE,RO]{\thepage}
\fancyhead[RE]{\textit{\nouppercase{\leftmark}}}
\fancyhead[LO]{\textit{\nouppercase{\rightmark}}}

\fancypagestyle{plain}{ %
\fancyhf{}
\renewcommand{\headrulewidth}{0pt}
\renewcommand{\footrulewidth}{0pt}}

% ------------------------------------------------------------------------
% Inne
% ------------------------------------------------------------------------
\frenchspacing
\setlength{\parskip}{3pt}           	%odst�p pomi�dzy akapitami
%\linespread{1.49}                    	%odst�p pomi�dzy liniami (interlinia)
\setcounter{tocdepth}{3}
\setcounter{secnumdepth}{3}


% ------------------------------------------------------------------------
% Polskie podpisy
% ------------------------------------------------------------------------
\renewcommand{\figurename}{Rys.}
\renewcommand{\tablename}{Tab.}

% ------------------------------------------------------------------------
% Bibliografia
% ------------------------------------------------------------------------
\bibliographystyle{unsrt}					% kolejno�� wed��g u�ycia
%\bibliographystyle{plain}					% kolejno�� alfabetyczna
  
  

%==========================================================================================
% Deklaracja fontow kapitalikowych z kodowaniem T1
%==========================================================================================
\DeclareFontShape{T1}{lmr}{bx}{sc} { <-> ssub * cmr/bx/sc }{}
\DeclareFontShape{T1}{lmr}{bx}{scit}{<-> ssub * cmr/bx/scsl}{}
%==========================================================================================
% Inne deklaracje
%==========================================================================================
