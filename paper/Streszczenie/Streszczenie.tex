\subsection*{Streszczenie}

G��wnym celem pracy jest przedstawienie wynik�w przeprowadzonych bada� dotycz�cych mo�liwo�ci zastosowania g��bokich sieci
neuronowych do wspomagania diagnostyki medycznej.
Przyk�adowym zadaniem, kt�re zosta�o wybrane jako obiekt bada� jest diagnozowanie stan�w padaczkowych na podstawie 
odczyt�w z elektroencefalogramu (EEG).

W pierwszej cz�ci pracy zawarte zosta�y informacje teoretyczne obejmuj�ce zagadnienia sztuczenej inteligencji, uczenia maszynowego oraz klasycznych i g��bokich sieci neuronowych. Przedstawiony zosta� r�wnie� wybrany problem praktyczny z dziedziny diagnostyki medycznej.

Druga cz�� przedstawia przegl�d narz�dzi programistycznych wykorzystywanych do implementacji g��bokich sieci neuronowych oraz implementacj� procesu nauki sieci neuronowej. Opisane zosta�y r�wnie� przeprowadzone badania maj�ce na celu sprawdzenie czy g��bokie sieci neuronowe mo�na zastosowa� do rozwi�zania wybranego problemu.

Motywacj� do podj�cia tematu pracy by�a ch�� rozpoznania popularnego w dzisiejszych czasach zagadnienia g��bokich sieci neuronowych oraz pr�ba zastosowania ich dla rzeczywistego problemu praktycznego.

\vspace{1cm}
\noindent\textbf{S�owa kluczowe:} Sieci neuronowe, diagnostyka medyczna, EEG, deep learning.
